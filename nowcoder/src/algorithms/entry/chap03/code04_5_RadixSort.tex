% Preamble
\documentclass[11pt]{article}
\title{RadixSort}
%\date{} 使title不显示默认时间格式,可以指定时间
\author{Johnny}
\setlength{\parskip}{10px}
% Packages
%\usepackage{amsmath}
\usepackage{geometry}
\geometry{left=2.0cm,right=2.0cm,top=2.5cm,bottom=3.5cm}

% Document
\begin{document}
    \maketitle
    \section{Conception}
    Radix sort belongs to "distributed sort", also known as the "bucket sort" or "bin sort".
    As the name implies, it distributes the elements to be sorted into certain "buckets" through partial information
    of the key value, so as to achieve the sorting effect.
    Radix sort algorithm is a sort of stability and its time complexity is $O(n\log_rm)$, where r is the radix taken,
    and m is the number of heaps.
    In some cases, the efficiency of the radix sort algorithm is higher than other ranking algorithms.

    \section{Algorithm Flow}
    \subsection{step 1.}
    Choose a number system like binary or decimal or octal as Radix based on the given array then add zeros in head to make
    all the numbers have the same digits.
    \subsection{step 2.}
    The corresponding number of buckets(containers) like array is allocated according to the taken Radix System.
    For example, if we choose decimal, the order of the buckets is 0 - 9.
    Then put all numbers into the buckets according to their least significant digit, which called Least Significant Digit first(LSD).
    Of course you can distribute them according to their most significant digit, but the sequence of array is from large to small,
    which called Most Significant Digit first(MSD).
    And the following steps, we take LSD as an example.
    \subsection{step 3.}
    Now, we put the buckets with the numbers in step 2 back into the original array according to the serial number of the bucket.
    Then we put all numbers into the buckets again according to the last digit of the Least Digit, which is very similar to step2.
    \subsection{step 4.}
    Repeat the above steps until we have finished the last loading and unloading which is depend on the most digit.
    And you will amazingly find that the numbers are already in order at this time.

    \section{Experiment}
    In the experiments, the version 1 uses decimal as the Radix and version 2 use binary as the base.\par
%    我们并没有显式的把所有数的位数补齐,而是以最大的数的位数为基准,直接取出他们的各个位数。对于那些位数不足的数,直接返回0即可
    \noindent In version1(\emph{code04\_RadixSortV1Decimal.java}), we didn't explicitly fill in the digits of all numbers,
    but took the digits of the largest one as a benchmark and directly took out their digits.For those numbers with
    insufficient digits, just return 0 directly.This step is mainly reflected in the Method \emph{getdDigits()}.

    %我是基于二进制,我本以为可以效仿版本1,只更改RADIX的值,但是出了很多错误,所以进行了更改。更好后的代码很简单,一看就懂。
    \noindent In version2(\emph{code05\_RadixSortV2Binary}), I'm based on binary.I thought I could imitate the version 1 and
    only change the value of RADIX,but there were many errors, so I changed it.The better code is very simple to understand.

    \noindent In version3(\emph{code06\_RadixSortV3Elegance}) which is taught ZuoShen used an array count to record the times
    of the current digit, instead of creating some containers.

    \section{Complexity Analysis}
    \subsection{Time Complexity}
    In Radix Sorting Alogrithm, set r to be the number of Radix Digit System, and m is the maximum number of digits after conversion
    based on the selected Radix Digit System, then its time complexity is $O(n\log_rm)$.
    For example, in code version1(\emph{code04\_RadixSortV1Decimal.java}), the Radix is decimal, and the maximum number is 9989,
    which is 4 digits.The the time complexity is $O(n\lg9989) \simeq O(n)$.This is the advantage of Radix Sort.
    %而在code v2中,r=2,这是显而易见的。最大数是89,但是并不意味着m=2,必须先把89转为2进制后,计算这个最大长度m
    In code version2(\emph{code05\_RadixSortV2Binary}) the r is 2,which is obvious.The maximum number is 89, but it doesn't mean
    that m is 2.You must covert 89 to binary to calculate the maximum length m(7).
    \subsection{Space Complexity}
    The Space Complexity of Radix Sort is $O(n)$.

    \section{Conclusion}
    %基数排序是一种稳定快速的排序算法。但是因为它通常要求数据是非负正数,而且比较浪费空间,因此使用并不广泛。
    %在基数为2进制的v2实验中,我陷入了一个误区,试图希望把一个数据全是二进制的数组使用基数排序方法排序后,得到一个有序数组,并且仍然是二进制的数据。
    %但是发现很难做到,因为我没有办法给一个参数变量加上二进制的前缀0b,它只能存在于常量。就是这样的一个问题耽误了好长时间。不过我学到了一个取一个数
    %二进制位数的好方法,那就是 num >> i & 1. 其中num是该数的十进制形式,i取值0,1,2... 代表该数的二进制从右往左取每一位数字。
    %V3版本是左神教的最优雅完美的写法,以后只用这个。
    Radix Sort is a stable and fast sorting algorithm.But because it usually requires data to be non-negative and positive and it wastes
    space, it is not widely used.
    In the code version2(\emph{code05\_RadixSortV2Binary}) experiment where the Radix System is binary, I fell into a misunderstanding.
    I tried to get an ordered after sorting an array with all binary data using the Radix Sort Method, and it was still binary number.
    But I found it difficult to do, because I have no way to add binary prefix '0b' to a parameter variable, it can only exist in constants.
    It is such a problem that delayed me for a long time.But I learned a new method to get the binary digits of a number, that is $num \gg i \& 1$,
    where num is the decimal form of the number and i takes the value 0,1,2... which represents the binary value of the number take each digit number
    from right to left.
    The code version3(\emph{code06\_RadixSortV3Elegance}) is the most elegant and perfect method taught by ZuoShen, and I will only use this in the future.

\end{document}